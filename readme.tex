% Created 2016-04-01 Fri 17:13
\documentclass[9pt,b5paper]{article}
\usepackage{graphicx}
\usepackage{xcolor}
\usepackage{xeCJK}
\setCJKmainfont{SimSun}
\usepackage{longtable}
\usepackage{float}
\usepackage{textcomp}
\usepackage{geometry}
\geometry{left=0cm,right=0cm,top=0cm,bottom=0cm}
\usepackage{multirow}
\usepackage{multicol}
\usepackage{listings}
\usepackage{algorithm}
\usepackage{algorithmic}
\usepackage{latexsym}
\usepackage{natbib}
\usepackage{fancyhdr}
\usepackage[xetex,colorlinks=true,CJKbookmarks=true,linkcolor=blue,urlcolor=blue,menucolor=blue]{hyperref}


\lstset{language=c++,numbers=left,numberstyle=\tiny,basicstyle=\ttfamily\small,tabsize=4,frame=none,escapeinside=``,extendedchars=false,keywordstyle=\color{blue!70},commentstyle=\color{red!55!green!55!blue!55!},rulesepcolor=\color{red!20!green!20!blue!20!}}
\author{deepwaterooo}
\date{\today}
\title{Tetris - Basic Implementation Practice for Android}
\hypersetup{
  pdfkeywords={},
  pdfsubject={},
  pdfcreator={Emacs 24.3.1 (Org mode 8.2.7c)}}
\begin{document}

\maketitle
\tableofcontents


\section{Better version, pretty good}
\label{sec-1}
\begin{itemize}
\item a video for this Tetris game can be directly watched at \url{https://www.youtube.com/watch?v=Ht4NOrEUtFk}
\item A video for the previous DrawingFun Android App can be watched at \url{https://www.youtube.com/watch?v=YV78Tk5--5M} , or by searching \textbf{deepwaterooo Wang}.
\item 
\item Starting my trial for OpenGL ES, need to figure out how to draw a game board.
\item Won't be able to work on it this weekend, but will work on it later on.
\item 
\item These video will serve as the indication that as an educated well practiced graduated student, I have the solid technological background, my problem solving skills, the spirit of implementing whatever ideas for apps that I feel I am capable, as well as confidence as an entry level mobile app programmer.
\item For the Tetris game, it's NOT the best product in my mind yet (though it is pretty good now and I will make it a my version of Tetris), but I want to record it so that more friends can enjoy the so far already achieved progress, and for those who just know me would be able to know what is my interested field.
\item By using SurfaceView who has a separate thread for drawing/painting, this game actually it pretty good already, at least should be about 80 out of 100.
\item Though I will continuous refine this game later on when I have time (Better version will be recorded and uploaded later within a month or so.), but I won't be able to work on it day in and day out recently, having other things occupied.
\end{itemize}

\section{其它类似游戏参考}
\label{sec-2}
\subsection{iTetris 俄罗斯方块}
\label{sec-2-1}
这是一款针对手机触屏的经典俄罗斯方块,可以选择传统及酷炫模式,用户可根据各自喜好自定义游戏背景。人性化的操作体验,让您重温儿时经典。该款游戏特征:

\begin{enumerate}
\item 尽量大的利用屏幕空间显示游戏,加大了游戏的可玩性
\item 支持触屏手势,同时支持虚拟按键
\item 虚拟方向按键盘初始化在右上角,用户可长按按键盘中心提起按键盘,拖动到自己任意觉得顺手的地方
\item 长按游戏空间任何地方可以隐藏、显示方向按键盘
\item 长按虚拟键盘中左中下键将发送连续按键信息,达到加速效果
\item 触摸游戏规则:点击屏幕改变方块形状;左右滑动改变方块左右位置,位置改变幅度与手指滑动速度与幅度有关,例如幅度较小滑动,方块将运行一格的位置,稍大滑动将运行两格位置,等等。向下加速也是这样。
\item 支持自定义游戏背景,如果有好的背景图片,用户可自行进入“菜单-游戏背景”里选择图片进行设置
\end{enumerate}

\section{References}
\label{sec-3}
\subsection{GLSurfaceView}
\label{sec-3-1}
\begin{itemize}
\item \url{http://hellosure.github.io/android/2015/06/01/android-glsurfaceview/}
\item \url{http://ju.outofmemory.cn/entry/172850}
\item 画图: \url{http://www.mobile-open.com/2015/81568.html}
\item \url{http://tangzm.com/blog/?p=20}
\item \url{http://www.apkbus.com/blog-99192-39584.html}
\item onDrawFrame intro: \url{http://www.jayway.com/2009/12/03/opengl-es-tutorial-for-android-part-i/}
\item failed: \url{http://stackoverflow.com/questions/28711850/android-opengl-how-to-draw-a-rectangle}
\item onTouchEvent: \url{http://blog.csdn.net/niu_gao/article/details/8673662}
\item volatile \url{http://www.voidcn.com/blog/fanfanxiaozu/article/p-3668133.html}
\item \url{http://mobile.51cto.com/aengine-437172.htm}
\item OpenGLES related: \url{http://stackoverflow.com/questions/9945321/triangle-opengl-in-android}
\item OpenGL ES 2.0 Sample Code: \url{http://androidbook.com/item/4254}
\item intros:详解 \url{http://blog.csdn.net/niu_gao/article/details/7566297}
\item 画线: \url{http://www.cnblogs.com/lhxin/archive/2012/06/01/2530828.html}
\item \url{http://bbs.9ria.com/thread-201740-1-1.html}
\item \url{http://imgtec.eetrend.com/blog/5078}
\item draw a ball \url{http://shikezhi.com/html/2015/android_1022/561912.html}
\item for Board c++: \url{http://www.jiancool.com/article/24471349949/}
\item possible? \url{http://code1.okbase.net/codefile/CCFormatter.java_2015072733469_393.htm}
\item \url{http://www.mobile-open.com/2015/80379.html}
\end{itemize}

\subsection{eventQueue vs SurfaceView threads}
\label{sec-3-2}
\begin{itemize}
\item Deeper summary, android graphics architecture: \url{http://hukai.me/android-deeper-graphics-architecture/}
\item 2 threads, load, read, \url{http://blog.csdn.net/hellogv/article/details/5986835}
\end{itemize}
\subsection{Canvas Path subclass}
\label{sec-3-3}
\begin{itemize}
\item how to define drawLine to be drawShapes?
\end{itemize}
\subsection{SurfaceView}
\label{sec-3-4}
\begin{itemize}
\item Surface runnable \url{http://android.okhelp.cz/surfaceview-implements-runnable-android-code/}
\item Example: \url{http://technicalsearch.iteye.com/blog/1967616}
\item \url{http://www.jcodecraeer.com/a/anzhuokaifa/androidkaifa/2012/1201/656.html}
\item Event Queue: \url{http://www.leestorm.com/post/17.html}
\item lockCanvas(Rect小区) \url{http://blog.csdn.net/alexander_xfl/article/details/13000347}
\item example: \url{http://fanli7.net/a/JAVAbiancheng/ANT/20120424/160203.html}
\item MotionEvent: \url{http://android.jobbole.com/82072/}
\item surfaceview双缓冲: \url{http://blog.csdn.net/cnbloger/article/details/7404485}
\item sth worth try: \url{http://www.lxway.com/969295592.htm}
\item Dont Understand: \url{http://blog.sina.com.cn/s/blog_5a6f39cf01012rtv.html}
\item tried: \url{http://bbs.csdn.net/topics/370074255} drawBitmap 2 canvas
\item slightly complicated: \url{http://www.lxway.com/148606691.htm}
\item slightly complicated: \url{http://www.lxway.com/186948856.htm}
\end{itemize}

\subsection{gestures}
\label{sec-3-5}
\begin{itemize}
\item \url{http://www.cnblogs.com/akira90/archive/2013/03/10/2952886.html}
\item Android 触摸手势基础 官方文档概览: \url{http://www.lxway.com/445554926.htm}
\item 手势: \url{http://wiki.jikexueyuan.com/project/material-design/patterns/gestures.html}
\item \url{http://www.lxway.com/601620614.htm}
\item \url{http://www.lxway.com/282219004.htm}
\item \url{http://www.lxway.com/906451412.htm}
\item \url{http://www.lxway.com/146619692.htm}
\item \url{http://www.lxway.com/4420294641.htm}
\item \url{http://www.lxway.com/155059816.htm}
\item \url{http://www.lxway.com/4019928952.htm}
\item 例子: \url{http://bbs.chinaunix.net/thread-3634477-1-1.html}
\item 例子: \url{http://www.bestappsmarket.com/p/app?appId=1192877&title=tetris-\%E4\%BF\%84\%E7\%BD\%97\%E6\%96\%AF\%E6\%96\%B9\%E5\%9D\%97}
\item 例子: \url{http://bbs.chinaunix.net/thread-3634477-1-1.html}

\item iTetris: \url{http://searchapp.soft4fun.net/article/information/iTetris\%20\%E4\%BF\%84\%E7\%BD\%97\%E6\%96\%AF\%E6\%96\%B9\%E5\%9D\%97/313319}
\item left right: \url{http://www.jb51.net/article/77028.htm}
\item AI: \url{http://www.cnblogs.com/youngshall/archive/2009/03/24/1420682.html}
\item 
\item 3/11/2016 Friday
\item \url{https://github.com/Almeros/android-gesture-detectors} mac
\item \url{http://www.jcodecraeer.com/a/anzhuokaifa/androidkaifa/2015/0211/2467.html}
\item \url{http://www.hejun.biz/81.html}
\item \url{http://www.jb51.net/article/38166.htm}
\item \url{http://www.jb51.net/article/37717.htm}
\item \url{http://mobile.51cto.com/aprogram-394841.htm}

\item TetrisBattle特殊轉入教學(Z S J L I)
\begin{itemize}
\item \url{https://www.youtube.com/watch?v=zW6Gp_7jl9I}
\end{itemize}
\item 推箱子: 第11章 Android游戏开发视频教程 益智游戏——推箱子
\begin{itemize}
\item \url{https://www.youtube.com/watch?v=glzxII1-P0A} 2.5D
\end{itemize}
\item 祖码游戏的设计与实现
\end{itemize}
% Emacs 24.3.1 (Org mode 8.2.7c)
\end{document}