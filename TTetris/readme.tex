% Created 2016-03-22 Tue 17:58
\documentclass[9pt,b5paper]{article}
\usepackage{graphicx}
\usepackage{xcolor}
\usepackage{xeCJK}
\setCJKmainfont{SimSun}
\usepackage{longtable}
\usepackage{float}
\usepackage{textcomp}
\usepackage{geometry}
\geometry{left=0cm,right=0cm,top=0cm,bottom=0cm}
\usepackage{multirow}
\usepackage{multicol}
\usepackage{listings}
\usepackage{algorithm}
\usepackage{algorithmic}
\usepackage{latexsym}
\usepackage{natbib}
\usepackage{fancyhdr}
\usepackage[xetex,colorlinks=true,CJKbookmarks=true,linkcolor=blue,urlcolor=blue,menucolor=blue]{hyperref}


\lstset{language=c++,numbers=left,numberstyle=\tiny,basicstyle=\ttfamily\small,tabsize=4,frame=none,escapeinside=``,extendedchars=false,keywordstyle=\color{blue!70},commentstyle=\color{red!55!green!55!blue!55!},rulesepcolor=\color{red!20!green!20!blue!20!}}
\author{deepwaterooo}
\date{\today}
\title{Tetris - Basic Implementation Practice for Android}
\hypersetup{
  pdfkeywords={},
  pdfsubject={},
  pdfcreator={Emacs 24.3.1 (Org mode 8.2.7c)}}
\begin{document}

\maketitle
\tableofcontents


\section{Debugging, Better version}
\label{sec-1}
\begin{itemize}
\item Fixed all the noticed carelessly-produced flying bugsss\ldots{}.. A basic game is in good shape now, will work on "polish" for viewing, sounds, efficiency during following days when I have time.
\item Has good chance/potential to make it a great game.
\item 
\item Recently sick, and last night got very bad sleep for about 4 hours (6am - 10am), not in very good condition now.
\item Because used SurfaceView threading, the game is much more smoother than the previous version, ideas are pretty straight forward, but still debugging\ldots{}
\item Android threading is something that I have barely tried before, but I will work on this one this time, hopefully will make this TTetris a fully functional game for major functionality.
\item Debugging, and production is on the way.
\end{itemize}

\section{其它类似游戏参考}
\label{sec-2}
\subsection{iTetris 俄罗斯方块}
\label{sec-2-1}

这是一款针对手机触屏的经典俄罗斯方块,可以选择传统及酷炫模式,用户可根据各自喜好自定义游戏背景。人性化的操作体验,让您重温儿时经典。

该款游戏特征:

1.尽量大的利用屏幕空间显示游戏,加大了游戏的可玩性
2.支持触屏手势,同时支持虚拟按键
\begin{itemize}
\item .虚拟方向按键盘初始化在右上角,用户可长按按键盘中心提起按键盘,拖动到自己任意觉得顺手的地方
\item .长按游戏空间任何地方可以隐藏、显示方向按键盘
\item .长按虚拟键盘中左中下键将发送连续按键信息,达到加速效果
\item .触摸游戏规则:点击屏幕改变方块形状;左右滑动改变方块左右位置,位置改变幅度与手指滑动速度与幅度有关,例如幅度较小滑动,方块将运行一格的位置,稍大滑动将运行两格位置,等等。向下加速也是这样。
\end{itemize}
3.支持自定义游戏背景,如果有好的背景图片,用户可自行进入“菜单-游戏背景”里选择图片进行设置

\section{手势定义}
\label{sec-3}

\section{References}
\label{sec-4}
\subsection{SurfaceView}
\label{sec-4-1}
\begin{itemize}
\item Example: \url{http://technicalsearch.iteye.com/blog/1967616}
\item \url{http://www.jcodecraeer.com/a/anzhuokaifa/androidkaifa/2012/1201/656.html}
\item Event Queue: \url{http://www.leestorm.com/post/17.html}
\item lockCanvas(Rect小区) \url{http://blog.csdn.net/alexander_xfl/article/details/13000347}
\item example: \url{http://fanli7.net/a/JAVAbiancheng/ANT/20120424/160203.html}
\item MotionEvent: \url{http://android.jobbole.com/82072/}
\item surfaceview双缓冲: \url{http://blog.csdn.net/cnbloger/article/details/7404485}
\item sth worth try: \url{http://www.lxway.com/969295592.htm}
\item Dont Understand: \url{http://blog.sina.com.cn/s/blog_5a6f39cf01012rtv.html}
\item tried: \url{http://bbs.csdn.net/topics/370074255} drawBitmap 2 canvas
\item slightly complicated: \url{http://www.lxway.com/148606691.htm}
\item slightly complicated: \url{http://www.lxway.com/186948856.htm}
\end{itemize}

\subsection{gestures}
\label{sec-4-2}
\begin{itemize}
\item \url{http://www.cnblogs.com/akira90/archive/2013/03/10/2952886.html}
\item Android 触摸手势基础 官方文档概览: \url{http://www.lxway.com/445554926.htm}
\item 手势: \url{http://wiki.jikexueyuan.com/project/material-design/patterns/gestures.html}
\item \url{http://www.lxway.com/601620614.htm}
\item \url{http://www.lxway.com/282219004.htm}
\item \url{http://www.lxway.com/906451412.htm}
\item \url{http://www.lxway.com/146619692.htm}
\item \url{http://www.lxway.com/4420294641.htm}
\item \url{http://www.lxway.com/155059816.htm}
\item \url{http://www.lxway.com/4019928952.htm}
\item 例子: \url{http://bbs.chinaunix.net/thread-3634477-1-1.html}
\item 例子: \url{http://www.bestappsmarket.com/p/app?appId=1192877&title=tetris-\%E4\%BF\%84\%E7\%BD\%97\%E6\%96\%AF\%E6\%96\%B9\%E5\%9D\%97}
\item 例子: \url{http://bbs.chinaunix.net/thread-3634477-1-1.html}

\item iTetris: \url{http://searchapp.soft4fun.net/article/information/iTetris\%20\%E4\%BF\%84\%E7\%BD\%97\%E6\%96\%AF\%E6\%96\%B9\%E5\%9D\%97/313319}
\item left right: \url{http://www.jb51.net/article/77028.htm}
\item AI: \url{http://www.cnblogs.com/youngshall/archive/2009/03/24/1420682.html}
\item 
\item 3/11/2016 Friday
\item \url{https://github.com/Almeros/android-gesture-detectors} mac
\item \url{http://www.jcodecraeer.com/a/anzhuokaifa/androidkaifa/2015/0211/2467.html}
\item \url{http://www.hejun.biz/81.html}
\item \url{http://www.jb51.net/article/38166.htm}
\item \url{http://www.jb51.net/article/37717.htm}
\item \url{http://mobile.51cto.com/aprogram-394841.htm}
\item 
\item 
\item 
\item 
\item 
\item 
\end{itemize}


\begin{itemize}
\item TetrisBattle特殊轉入教學(Z S J L I)
\begin{itemize}
\item \url{https://www.youtube.com/watch?v=zW6Gp_7jl9I}
\end{itemize}
\item 推箱子: 第11章 Android游戏开发视频教程 益智游戏——推箱子
\begin{itemize}
\item \url{https://www.youtube.com/watch?v=glzxII1-P0A} 2.5D
\end{itemize}
\item 祖码游戏的设计与实现
\end{itemize}
% Emacs 24.3.1 (Org mode 8.2.7c)
\end{document}