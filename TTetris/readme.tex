% Created 2016-03-29 Tue 18:11
\documentclass[9pt,b5paper]{article}
\usepackage{graphicx}
\usepackage{xcolor}
\usepackage{xeCJK}
\setCJKmainfont{SimSun}
\usepackage{longtable}
\usepackage{float}
\usepackage{textcomp}
\usepackage{geometry}
\geometry{left=0cm,right=0cm,top=0cm,bottom=0cm}
\usepackage{multirow}
\usepackage{multicol}
\usepackage{listings}
\usepackage{algorithm}
\usepackage{algorithmic}
\usepackage{latexsym}
\usepackage{natbib}
\usepackage{fancyhdr}
\usepackage[xetex,colorlinks=true,CJKbookmarks=true,linkcolor=blue,urlcolor=blue,menucolor=blue]{hyperref}


\lstset{language=c++,numbers=left,numberstyle=\tiny,basicstyle=\ttfamily\small,tabsize=4,frame=none,escapeinside=``,extendedchars=false,keywordstyle=\color{blue!70},commentstyle=\color{red!55!green!55!blue!55!},rulesepcolor=\color{red!20!green!20!blue!20!}}
\author{deepwaterooo}
\date{\today}
\title{Tetris - Basic Implementation Practice for Android}
\hypersetup{
  pdfkeywords={},
  pdfsubject={},
  pdfcreator={Emacs 24.3.1 (Org mode 8.2.7c)}}
\begin{document}

\maketitle
\tableofcontents


\section{Debugging, Better version}
\label{sec-1}
\begin{itemize}
\item Projection is basically smooth now, this is pretty much all the features that I have ever thought and wanted to implement, except the queueEvent GLSurfaceView vs threads left to be the big challenge (or if after all details having been done, if the game IS REALLY smooth, I may not need implement either of those then).
\item I may need to think how to organize the different objects. For projection related methods, there maybe better way to code then what I have right now, but this is the trial implementation, and I need to think about them.
\item After game in shape for projection, will rethink about motionevents, or GLSurfaceView for queueEvent.
\item will try to rewirte motion event a little bit (slightly change touch motionevent definitions according to previous DrawingFun app), if doesn't work as expected, then try GLSurfaceView queueEvent.
\item 
\item apply MotionEvent queue, use either SurfaceView with multiple threads (if possible) or GLSurfaceView.
\item apply sound effects to make the game more fun.
\item The above 2, either one goes first.
\item One more "Hold" choice besides the Next Rect for peek, lower priority.
\item Noticed more bugs, but I am currently focusing on make the game super.
\item Fixed minor mindless codes when I coded with headache.
\item 
\item Fixed all the noticed carelessly-produced flying bugsss\ldots{}.. A basic game is in good shape now, will work on "polish" for viewing, sounds, efficiency during following days when I have time.
\item Has good chance/potential to make it a great game with Event Queue for motion events.
\item Recently sick, and last night got very bad sleep for about 4 hours (6am - 10am), not in very good condition now.
\item Because used SurfaceView threading, the game is much more smoother than the previous version, ideas are pretty straight forward, but still debugging\ldots{}
\item Android threading is something that I have barely tried before, but I will work on this one this time, hopefully will make this TTetris a fully functional game for major functionality.
\item Debugging, and production is on the way.
\end{itemize}

\section{其它类似游戏参考}
\label{sec-2}
\subsection{iTetris 俄罗斯方块}
\label{sec-2-1}
这是一款针对手机触屏的经典俄罗斯方块,可以选择传统及酷炫模式,用户可根据各自喜好自定义游戏背景。人性化的操作体验,让您重温儿时经典。该款游戏特征:

\begin{enumerate}
\item 尽量大的利用屏幕空间显示游戏,加大了游戏的可玩性
\item 支持触屏手势,同时支持虚拟按键
\item 虚拟方向按键盘初始化在右上角,用户可长按按键盘中心提起按键盘,拖动到自己任意觉得顺手的地方
\item 长按游戏空间任何地方可以隐藏、显示方向按键盘
\item 长按虚拟键盘中左中下键将发送连续按键信息,达到加速效果
\item 触摸游戏规则:点击屏幕改变方块形状;左右滑动改变方块左右位置,位置改变幅度与手指滑动速度与幅度有关,例如幅度较小滑动,方块将运行一格的位置,稍大滑动将运行两格位置,等等。向下加速也是这样。
\item 支持自定义游戏背景,如果有好的背景图片,用户可自行进入“菜单-游戏背景”里选择图片进行设置
\end{enumerate}

\section{References}
\label{sec-3}
\subsection{eventQueue vs SurfaceView threads}
\label{sec-3-1}
\begin{itemize}
\item Deeper summary, android graphics architecture: \url{http://hukai.me/android-deeper-graphics-architecture/}
\item 2 threads, load, read, \url{http://blog.csdn.net/hellogv/article/details/5986835}
\end{itemize}
\subsection{Canvas Path subclass}
\label{sec-3-2}
\begin{itemize}
\item how to define drawLine to be drawShapes?
\end{itemize}
\subsection{SurfaceView}
\label{sec-3-3}
\begin{itemize}
\item Surface runnable \url{http://android.okhelp.cz/surfaceview-implements-runnable-android-code/}
\item Example: \url{http://technicalsearch.iteye.com/blog/1967616}
\item \url{http://www.jcodecraeer.com/a/anzhuokaifa/androidkaifa/2012/1201/656.html}
\item Event Queue: \url{http://www.leestorm.com/post/17.html}
\item lockCanvas(Rect小区) \url{http://blog.csdn.net/alexander_xfl/article/details/13000347}
\item example: \url{http://fanli7.net/a/JAVAbiancheng/ANT/20120424/160203.html}
\item MotionEvent: \url{http://android.jobbole.com/82072/}
\item surfaceview双缓冲: \url{http://blog.csdn.net/cnbloger/article/details/7404485}
\item sth worth try: \url{http://www.lxway.com/969295592.htm}
\item Dont Understand: \url{http://blog.sina.com.cn/s/blog_5a6f39cf01012rtv.html}
\item tried: \url{http://bbs.csdn.net/topics/370074255} drawBitmap 2 canvas
\item slightly complicated: \url{http://www.lxway.com/148606691.htm}
\item slightly complicated: \url{http://www.lxway.com/186948856.htm}
\end{itemize}

\subsection{gestures}
\label{sec-3-4}
\begin{itemize}
\item \url{http://www.cnblogs.com/akira90/archive/2013/03/10/2952886.html}
\item Android 触摸手势基础 官方文档概览: \url{http://www.lxway.com/445554926.htm}
\item 手势: \url{http://wiki.jikexueyuan.com/project/material-design/patterns/gestures.html}
\item \url{http://www.lxway.com/601620614.htm}
\item \url{http://www.lxway.com/282219004.htm}
\item \url{http://www.lxway.com/906451412.htm}
\item \url{http://www.lxway.com/146619692.htm}
\item \url{http://www.lxway.com/4420294641.htm}
\item \url{http://www.lxway.com/155059816.htm}
\item \url{http://www.lxway.com/4019928952.htm}
\item 例子: \url{http://bbs.chinaunix.net/thread-3634477-1-1.html}
\item 例子: \url{http://www.bestappsmarket.com/p/app?appId=1192877&title=tetris-\%E4\%BF\%84\%E7\%BD\%97\%E6\%96\%AF\%E6\%96\%B9\%E5\%9D\%97}
\item 例子: \url{http://bbs.chinaunix.net/thread-3634477-1-1.html}

\item iTetris: \url{http://searchapp.soft4fun.net/article/information/iTetris\%20\%E4\%BF\%84\%E7\%BD\%97\%E6\%96\%AF\%E6\%96\%B9\%E5\%9D\%97/313319}
\item left right: \url{http://www.jb51.net/article/77028.htm}
\item AI: \url{http://www.cnblogs.com/youngshall/archive/2009/03/24/1420682.html}
\item 
\item 3/11/2016 Friday
\item \url{https://github.com/Almeros/android-gesture-detectors} mac
\item \url{http://www.jcodecraeer.com/a/anzhuokaifa/androidkaifa/2015/0211/2467.html}
\item \url{http://www.hejun.biz/81.html}
\item \url{http://www.jb51.net/article/38166.htm}
\item \url{http://www.jb51.net/article/37717.htm}
\item \url{http://mobile.51cto.com/aprogram-394841.htm}

\item TetrisBattle特殊轉入教學(Z S J L I)
\begin{itemize}
\item \url{https://www.youtube.com/watch?v=zW6Gp_7jl9I}
\end{itemize}
\item 推箱子: 第11章 Android游戏开发视频教程 益智游戏——推箱子
\begin{itemize}
\item \url{https://www.youtube.com/watch?v=glzxII1-P0A} 2.5D
\end{itemize}
\item 祖码游戏的设计与实现
\end{itemize}
% Emacs 24.3.1 (Org mode 8.2.7c)
\end{document}